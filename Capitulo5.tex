\section{Descripción del problema}

La comunicación es un aspecto fundamental de la interacción humana, y su falta puede resultar en un aislamiento significativo. Las personas sordas enfrentan desafíos únicos en su capacidad para comunicarse con quienes no dominan el lenguaje de señas. A pesar de los avances en la inclusión social, el acceso a la comunicación efectiva sigue siendo un obstáculo para la comunidad sorda. El lenguaje de señas es la principal forma de comunicación para millones de personas en todo el mundo, y su uso no es universalmente reconocido o comprendido por la población oyente. Esta desconexión crea barreras que impiden la interacción social fluida y pueden contribuir a la marginación de las personas sordas en diversas esferas de la vida, incluyendo la educación, el empleo y el acceso a servicios de salud.

La falta de comprensión del lenguaje de señas por parte de las personas oyentes no solo limita la comunicación, sino que también puede generar malentendidos, frustraciones y situaciones de exclusión. Además, en situaciones críticas, como emergencias médicas o desastres naturales, la falta de un medio efectivo para comunicar información vital puede tener consecuencias graves. A menudo, las personas sordas dependen de intérpretes para facilitar la comunicación, pero la disponibilidad de estos profesionales puede ser limitada, lo que agrava la situación. Este hecho resalta la necesidad de soluciones tecnológicas que no solo proporcionen traducción en tiempo real, sino que también sean accesibles y fáciles de usar en la vida diaria.

Los avances en tecnología, particularmente en el ámbito de la inteligencia artificial y los dispositivos portátiles, han abierto nuevas oportunidades para abordar estos problemas. Los guantes inteligentes equipados con sensores flexibles y algoritmos de aprendizaje automático representan una solución prometedora para la traducción del lenguaje de señas. Sin embargo, a pesar de su potencial, la implementación de estas tecnologías enfrenta varios desafíos. La precisión en la detección de gestos, la adaptabilidad a diferentes estilos de comunicación y la aceptación social son solo algunos de los aspectos que deben ser considerados. 

El desarrollo de un sistema que sea práctico y asequible para la comunidad sorda es crucial. La efectividad de cualquier dispositivo dependerá no solo de su funcionalidad técnica, sino también de su aceptación por parte de los usuarios. Esto implica la necesidad de realizar estudios de usabilidad y considerar el feedback de la comunidad sorda durante el proceso de diseño y desarrollo.

La falta de una solución efectiva para la traducción del lenguaje de señas representa un problema significativo que limita la interacción y la inclusión social de las personas sordas. A medida que la tecnología avanza, es esencial explorar y desarrollar dispositivos que no solo superen las barreras de comunicación, sino que también promuevan la equidad y la inclusión en todos los aspectos de la vida. Esta investigación se centra en abordar estos desafíos, con el objetivo de crear un sistema que no solo traduzca el lenguaje de señas de manera efectiva, sino que también fomente una mayor comprensión y empatía entre la comunidad sorda y la población oyente.
