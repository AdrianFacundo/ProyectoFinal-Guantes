\section{Antecedentes teóricos}
La detección y traducción del lenguaje de señas ha sido un tema de investigación y desarrollo en el ámbito de la tecnología y la inclusión social durante varias décadas. A medida que la tecnología ha avanzado, se han explorado diversas metodologías y herramientas para facilitar la comunicación entre personas sordas y oyentes. Este apartado revisa algunos de los trabajos más relevantes en este campo, centrándose en los avances tecnológicos, las aplicaciones de aprendizaje automático, y las iniciativas que han surgido para abordar el problema de la comunicación en la comunidad sordomuda.

\begin{itemize} 
\item \textbf{Tecnología de Detección de Gestos} \newline
Se centra en interpretar y reconocer los movimientos humanos, siendo fundamental para la traducción del lenguaje de señas. Utiliza diversas metodologías, como la visión por computadora, que analiza imágenes y videos para identificar gestos, y sensores inerciales que capturan el movimiento de las extremidades. Los algoritmos de aprendizaje automático, incluyendo redes neuronales profundas, permiten clasificar gestos a partir de grandes conjuntos de datos, mejorando la precisión del reconocimiento. Sin embargo, enfrenta desafíos como la variabilidad entre usuarios y las condiciones ambientales, lo que exige soluciones robustas para su aplicación en tiempo real. Esta tecnología tiene aplicaciones en interfaces de usuario, asistencia a personas con discapacidades y mejora de la interacción humano-computadora, lo que la convierte en un área de investigación en constante evolución. \newline
Referencia: Koller, O., et al. (2015). "Real-time Gesture Recognition: A Survey". \cite{core_ac_153562887}

\item \textbf{Wearables y Sensores} \newline 
Los dispositivos portátiles han ganado popularidad en el ámbito de la tecnología de detección de gestos, especialmente en el contexto de la traducción del lenguaje de señas. Un ejemplo destacado es el uso de guantes equipados con sensores flexibles que capturan los movimientos de los dedos y las manos. Estos guantes utilizan sensores inerciales y de flexión para detectar la posición y el movimiento de los dedos, permitiendo una traducción más precisa de los gestos en comparación con los sistemas basados en visión. La principal ventaja de los guantes es su capacidad para funcionar en entornos variados sin la necesidad de cámaras, aumentando así su versatilidad y aplicabilidad. Investigaciones como las realizadas por Ranjan et al. (2018) han demostrado que estos guantes pueden detectar gestos en tiempo real, proporcionando una interfaz intuitiva para la comunicación entre personas sordas y oyentes. Este enfoque no solo mejora la calidad de la traducción, sino que también promueve la inclusión social al facilitar la comunicación directa en diversas situaciones de la vida cotidiana.
\newline
Referencia: Ranjan, R., et al. (2018). "A Wearable Glove for Gesture Recognition". \cite{mdpi_sensors_18_367}

\item \textbf{Aprendizaje Automático y Reconocimiento de Patrones} \newline 
El aprendizaje automático ha revolucionado la detección y traducción del lenguaje de señas, ofreciendo soluciones más precisas y adaptables. En el estudio de Wang et al. (2019), se exploran las técnicas de redes neuronales profundas para el reconocimiento de patrones en secuencias de gestos. Estas redes son entrenadas utilizando grandes conjuntos de datos que representan diversos gestos del lenguaje de señas, lo que permite al sistema no solo reconocer gestos individuales, sino también identificar combinaciones y estructuras complejas del lenguaje. Esta metodología ha demostrado mejorar significativamente la precisión de la traducción, lo cual es crucial para una comunicación efectiva entre personas sordas y oyentes. La capacidad de estos modelos para aprender y generalizar a partir de datos variados hace que sean una herramienta fundamental en la búsqueda de soluciones innovadoras para la inclusión social. \newline
Referencia: Wang, Y., et al. (2019). "Deep Learning Techniques for Sign Language Recognition". \cite{mdpi_sensors_18_367}

\item \textbf{Proyectos de Investigación y Aplicaciones Prácticas} \newline 
El proyecto "Sign Language Recognition using Smart Gloves" desarrollado por un equipo de investigadores en la Universidad de Ciudad del Cabo se centra en la creación de guantes inteligentes equipados con sensores para la detección y traducción del lenguaje de señas. Este sistema utiliza sensores de flexión y acelerómetros para captar los movimientos de las manos y los dedos, permitiendo la traducción en tiempo real de los gestos en texto o audio. La interfaz de usuario facilita la comunicación entre personas sordas y oyentes, mejorando la accesibilidad y la inclusión social. Este tipo de proyectos subraya la importancia de la tecnología portátil en la eliminación de barreras comunicativas y en el fomento de una mayor interacción social. \newline 
Referencia: "Sign Language Recognition using Smart Gloves". Universidad de Ciudad del Cabo. \cite{researchgate_sign_language}

\item \textbf{Detección y Traducción de Gestos} \newline 
El artículo "Hand Gesture Recognition Using Machine Learning Techniques: A Review" de Khan et al. (2019) presenta un análisis exhaustivo de diversas técnicas de reconocimiento de gestos de la mano utilizando enfoques de aprendizaje automático. El estudio examina los algoritmos más utilizados, como máquinas de soporte vectorial (SVM), redes neuronales y métodos basados en características, destacando sus ventajas y desventajas en el contexto del reconocimiento de gestos. Además, se discuten los desafíos asociados con la variabilidad de los gestos, el ruido en los datos y la necesidad de conjuntos de datos robustos para el entrenamiento de modelos. Este trabajo subraya la importancia de desarrollar sistemas precisos y eficientes para la traducción de gestos, lo cual es crucial para aplicaciones como la interpretación del lenguaje de señas. \newline 
Referencia: Khan, A. I., et al. (2019). "Hand Gesture Recognition Using Machine Learning Techniques: A Review". \cite{sciencedirect_s187705091931896x}

\item \textbf{Sistemas Basados en Sensores} \newline 
El artículo "Real-time Sign Language Recognition Using Flexible Sensors" de Yang et al. (2020) aborda el desarrollo de un sistema para el reconocimiento de lenguaje de señas en tiempo real utilizando sensores flexibles. Este estudio explora cómo los sensores flexibles pueden capturar los movimientos de las manos y los dedos, proporcionando una solución más precisa y accesible en comparación con los métodos basados en cámaras. Se detalla la integración de estos sensores en dispositivos portátiles, lo que permite una mayor libertad de movimiento para los usuarios. El trabajo destaca la efectividad de este enfoque en entornos del mundo real, así como su potencial para mejorar la comunicación entre personas sordas y oyentes. \newline 
Referencia: Yang, Z., et al. (2020). "Real-time Sign Language Recognition Using Flexible Sensors". \cite{mdpi_sensors_19_2295}


\item \textbf{Aprendizaje Profundo en Lenguaje de Señas} \newline 
El artículo "Deep Learning for Sign Language Recognition: A Review" de Ghosh et al. (2020) presenta una revisión exhaustiva de las técnicas de aprendizaje profundo aplicadas al reconocimiento de lenguaje de señas. Los autores discuten diversas arquitecturas de redes neuronales, como las redes convolucionales (CNN) y las redes recurrentes (RNN), que se utilizan para mejorar la precisión en la detección y traducción de gestos. El estudio resalta cómo el aprendizaje profundo ha revolucionado la forma en que se aborda el reconocimiento de gestos, permitiendo a los modelos aprender de grandes volúmenes de datos y adaptarse a la variabilidad del lenguaje de señas. Además, se analizan los desafíos existentes y se proponen direcciones futuras para la investigación en este campo.  \newline 
Referencia: Ghosh, S., et al. (2020). "Deep Learning for Sign Language Recognition: A Review". \cite{mdpi_applied_sci_10_2458}

\item \textbf{Dispositivos Inteligentes y Tecnología Inalámbrica} \newline 
El artículo "A Smart Glove for Sign Language Recognition: Design and Development" de Sadiq et al. (2021) explora el diseño y desarrollo de un guante inteligente para el reconocimiento de lenguaje de señas. Los autores presentan un sistema que integra tecnología inalámbrica para facilitar la transmisión de datos en tiempo real, lo que mejora la usabilidad y la eficiencia en la comunicación. Se discuten los componentes del guante, incluyendo sensores flexibles y módulos de comunicación, así como el proceso de calibración y entrenamiento del sistema de reconocimiento. Este trabajo destaca la importancia de combinar dispositivos inteligentes con tecnología inalámbrica para crear soluciones accesibles y efectivas en la traducción del lenguaje de señas.  \newline 
Referencia: Sadiq, M. B., et al. (2021). "A Smart Glove for Sign Language Recognition: Design and Development". \cite{sciencedirect_s0957417421003214}

\item \textbf{Proyectos Colaborativos y Comunitarios} \newline 
El artículo "Community-Driven Approaches to Sign Language Recognition: An Insight" de Rodriguez et al. (2021) examina enfoques impulsados por la comunidad para el reconocimiento del lenguaje de señas. Los autores destacan la importancia de involucrar a las comunidades sordas en el proceso de desarrollo de tecnologías de reconocimiento, argumentando que su participación activa no solo mejora la precisión de los sistemas, sino que también garantiza que las soluciones se alineen con las necesidades y deseos de los usuarios finales. Este estudio ilustra varios proyectos donde la colaboración entre investigadores, desarrolladores y comunidades ha llevado a innovaciones significativas en la detección y traducción del lenguaje de señas, promoviendo un sentido de propiedad y empoderamiento en las comunidades afectadas. \newline  
Referencia: Rodriguez, R., et al. (2021). "Community-Driven Approaches to Sign Language Recognition: An Insight". \cite{journaloflanguageandlinguisticstudies}

\end{itemize}
