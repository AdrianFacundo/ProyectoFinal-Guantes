\section{Introducción}
La comunicación es una de las actividades más importantes en la vida de cualquier persona, ya que nos permite expresar ideas, sentimientos y conectar con nuestro entorno. Para la mayoría, el lenguaje hablado es el principal medio de interacción, pero para las personas sordas o con dificultades auditivas, la realidad es diferente. En su caso, el lenguaje de señas es la herramienta esencial para poder comunicarse con los demás. Aunque este sistema es completo y eficaz dentro de la comunidad sorda, su principal desafío es que no es ampliamente comprendido por el resto de la población. Esta barrera en la comunicación entre personas sordas y oyentes genera situaciones de incomprensión, que a menudo terminan aislando a quienes dependen del lenguaje de señas. 

El problema de la falta de comunicación efectiva no solo afecta la vida cotidiana, sino también situaciones críticas en sectores como la educación, la atención médica y el ámbito laboral. Si bien existen soluciones tecnológicas como aplicaciones de traducción o servicios de interpretación, no siempre son accesibles o prácticas, ya que muchas dependen de un tercero para su funcionamiento. En este caso se muestra la necesidad de soluciones más innovadoras que faciliten una comunicación fluida sin depender de otros.

Es en este contexto que surge la idea de los guantes traductores de lenguaje de señas, una tecnología diseñada para facilitar la comunicación entre personas sordas y oyentes de manera directa, rápida y eficiente. El proyecto tiene como objetivo desarrollar guantes equipados con sensores que puedan detectar los movimientos de las manos y los dedos, traduciendo estos gestos en tiempo real a texto o voz. Los guantes contarán con sensores flexibles para medir la curvatura de los dedos, así como sensores giroscópicos y acelerómetros que registrarán el movimiento y la orientación de las manos. Todo esto se conectará a través de microcontroladores ESP32, los cuales permitirán procesar y transmitir los datos de manera eficiente.

Una de las principales ventajas de este proyecto es su enfoque en la portabilidad y facilidad de uso. Los guantes estarán diseñados para ser inalámbricos, utilizando conexiones Wi-Fi para transmitir los datos a un dispositivo receptor, donde se llevará a cabo la traducción del lenguaje de señas a voz o texto. Este enfoque aumenta la libertad de movimiento y la comodidad para el usuario, permitiendo que el sistema se utilice en cualquier lugar y momento, sin la necesidad de depender de intérpretes o equipos costosos.

Se planea implementar técnicas avanzadas de aprendizaje automático en el sistema para mejorar tanto la precisión como la adaptabilidad de los guantes traductores de lenguaje de señas. Se utilizará Tiny Machine Learning (TinyML), una rama del aprendizaje automático que se enfoca en modelos ligeros y eficientes, capaces de ejecutarse directamente en dispositivos pequeños y de bajo consumo, como los microcontroladores que controlan los guantes. La incorporación de TinyML permitirá que los guantes no solo realicen la traducción de señas en tiempo real, sino que también aprendan y se ajusten de manera continua a las particularidades del usuario, mejorando su rendimiento a lo largo del tiempo.

El proceso de aprendizaje automático que se integrará en los guantes se basa en el uso de un conjunto de datos específicos de lenguaje de señas. Estos datos consisten en una amplia variedad de gestos y movimientos que las personas sordas utilizan comúnmente para comunicarse. A través de estos datos, el sistema podrá crear modelos de reconocimiento de patrones que detectarán de manera precisa los gestos realizados por el usuario, traduciéndolos a texto o voz.

Uno de los beneficios clave de esta capacidad de aprendizaje es que el sistema podrá adaptarse a las variaciones individuales de cada usuario. Cada persona tiene su propio estilo al realizar señas, que puede variar ligeramente en velocidad, fuerza y ángulo de los movimientos. Esta personalización es crucial, ya que no solo aumenta la precisión de la traducción, sino que también reduce la frustración del usuario al sentir que sus señas están siendo malinterpretadas por el sistema.

El uso de TinyML también contribuye a mejorar la escalabilidad del sistema. Al implementar un enfoque de aprendizaje automático, los guantes podrían ser utilizados en una amplia variedad de contextos sin necesidad de modificaciones significativas en su hardware. Por ejemplo, se podrían personalizar para adaptarse a diferentes lenguas de señas de distintos países, simplemente entrenando el sistema con los gestos y signos correspondientes a cada región. Del mismo modo, podrían integrarse con tecnologías adicionales, como dispositivos de asistencia auditiva, plataformas de comunicación en línea, o aplicaciones móviles, lo que ampliaría aún más su funcionalidad y su impacto en la vida de las personas sordas.

Esta capacidad de aprendizaje continuo no solo hace que el dispositivo sea más útil y adaptable, sino que también lo convierte en una herramienta más inclusiva. El hecho de que el sistema pueda mejorar y adaptarse según las necesidades de cada usuario refuerza la idea de que la tecnología debe ser accesible para todos, independientemente de las diferencias individuales.

Desde un punto de vista técnico, el uso de sensores flexibles y giroscopios permite captar con precisión cada movimiento, lo cual es fundamental para interpretar correctamente el lenguaje de señas, ya que cualquier variación en los gestos puede cambiar el significado. Además, el ESP32 es un microcontrolador potente y económico que permite procesar grandes cantidades de datos en tiempo real, asegurando que la traducción de los gestos sea rápida y eficiente.

A largo plazo, este tipo de soluciones puede generar cambios significativos en la percepción social de la discapacidad auditiva. Los guantes traductores no solo facilitan la comunicación, sino que también envían un mensaje importante sobre la importancia de crear tecnologías accesibles que incluyan a todos. Esto es fundamental en una sociedad que, aunque ha avanzado en términos de inclusión, todavía enfrenta muchos desafíos para garantizar la equidad en la participación de personas con discapacidades. Así, este proyecto no solo tiene un valor técnico, sino también un profundo impacto social, promoviendo un futuro en el que la inclusión y la accesibilidad sean derechos garantizados para todos.