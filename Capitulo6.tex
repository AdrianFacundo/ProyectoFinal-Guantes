\section{Motivación}

La motivación detrás de este proyecto surge de un deseo de contribuir a la sociedad y mejorar la calidad de vida de las personas sordas. En un mundo donde la comunicación es la piedra angular de las interacciones humanas, es inaceptable que una parte significativa de la población se vea limitada por barreras lingüísticas que dificultan su acceso a la información, la educación y la participación plena en la sociedad. A través del desarrollo de guantes inteligentes para la detección y traducción del lenguaje de señas, buscamos no solo ofrecer una solución tecnológica, sino también fomentar una cultura de inclusión y respeto hacia las personas sordas.

El impacto positivo que este proyecto puede tener va más allá de la mera traducción de gestos. Al eliminar la barrera del lenguaje, se facilita la comunicación directa entre personas sordas y oyentes, promoviendo la interacción social y el entendimiento mutuo. Esto no solo mejora la calidad de vida de los usuarios, sino que también enriquece la comunidad en su conjunto al fomentar la diversidad y la empatía. Cada pequeño paso hacia la inclusión contribuye a construir una sociedad más justa y equitativa, donde todos tengan la oportunidad de participar y expresar sus ideas y sentimientos sin restricciones.

Este proyecto tiene el potencial de inspirar a otros a reconocer y abordar las necesidades de la comunidad sorda. Al demostrar que la tecnología puede ser una herramienta poderosa para la inclusión, se abre la puerta a nuevas iniciativas y colaboraciones que pueden transformar la forma en que interactuamos con la diversidad lingüística y cultural. La motivación de este trabajo es, en última instancia, ser un agente de cambio y contribuir a un futuro en el que la comunicación sea accesible para todos, sin importar sus capacidades auditivas. Así, se espera que el desarrollo de estos guantes inteligentes no solo sirva como un avance técnico, sino también como un símbolo de esperanza y progreso hacia una sociedad más inclusiva.
