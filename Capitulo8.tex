\section{Alcances y Limitaciones}

El proyecto de desarrollo de guantes inteligentes para la detección y traducción del lenguaje de señas tiene un alcance definido que incluye varias dimensiones. En primer lugar, se compromete a diseñar, prototipar y probar un dispositivo que permita la traducción en tiempo real de gestos del lenguaje de señas. Este proceso abarcará la integración de sensores flexibles y tecnologías de aprendizaje automático para mejorar la precisión y adaptabilidad del sistema. Además, se prevé la creación de una interfaz de usuario que facilite la interacción con el dispositivo, permitiendo a los usuarios sordos y oyentes comunicarse de manera efectiva.

El proyecto también enfrenta ciertas limitaciones que podrían afectar el cumplimiento de los objetivos planteados. Una de las principales limitaciones es la disponibilidad y calidad del material utilizado en la fabricación de los guantes. Los sensores y componentes electrónicos necesarios para lograr un rendimiento óptimo pueden tener restricciones en cuanto a costo, disponibilidad y calidad. La precisión del sistema dependerá en gran medida de la calidad de estos materiales, lo que podría influir en la eficacia de la traducción del lenguaje de señas.

La complejidad inherente al reconocimiento de gestos es otro factor limitante. El lenguaje de señas es dinámico y varía entre diferentes culturas y contextos, lo que puede dificultar la creación de un modelo de aprendizaje automático que reconozca todos los signos y combinaciones posibles. El entrenamiento del modelo requerirá un conjunto de datos diverso y representativo, lo que puede ser un desafío en términos de recopilación y validación de datos.

El tiempo y los recursos disponibles para la investigación y el desarrollo del proyecto también son factores limitantes. La construcción de un prototipo funcional y su posterior prueba en entornos reales requerirá una planificación cuidadosa y una asignación adecuada de recursos, tanto humanos como materiales. Es importante ser realista acerca de estos desafíos y mantener un enfoque flexible que permita adaptarse a las circunstancias cambiantes a lo largo del desarrollo del proyecto.
