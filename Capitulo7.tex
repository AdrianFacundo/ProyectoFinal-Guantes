\section{Justificación}

La necesidad de desarrollar guantes inteligentes para la detección y traducción del lenguaje de señas se justifica en múltiples dimensiones. En primer lugar, la comunidad sorda enfrenta desafíos significativos en su vida diaria debido a las barreras de comunicación. A menudo, se ven excluidos de interacciones sociales, oportunidades laborales y acceso a servicios esenciales, lo que resulta en un sentimiento de aislamiento y frustración. La implementación de esta tecnología innovadora busca mitigar esos desafíos, proporcionando a las personas sordas una herramienta que les permita comunicarse de manera más efectiva y fluida con quienes los rodean. Este enfoque no solo responde a una necesidad práctica, sino que también promueve un cambio cultural hacia la aceptación y el respeto de la diversidad.

La justificación técnica del proyecto radica en los avances recientes en tecnologías de sensores, aprendizaje automático y dispositivos portátiles. Estos desarrollos permiten crear soluciones más precisas y accesibles que nunca. Al integrar estas tecnologías en un solo dispositivo, se maximiza la eficiencia del sistema, lo que resulta en una traducción de lenguaje de señas más precisa y en tiempo real. Esto es fundamental, ya que la comunicación efectiva no solo se basa en la traducción literal de gestos, sino también en la interpretación del contexto y la fluidez del lenguaje. 

Por último, la iniciativa tiene el potencial de servir como un modelo para futuras investigaciones y desarrollos en el campo de la inclusión social y la tecnología. La creación de guantes inteligentes no solo beneficiará a la comunidad sorda, sino que también puede inspirar a otros investigadores y desarrolladores a explorar soluciones innovadoras para diversos problemas de comunicación y accesibilidad. Al abordar esta problemática de manera integral y comprometida, se busca no solo ofrecer una solución técnica, sino también contribuir a una sociedad más inclusiva y equitativa donde todas las voces sean escuchadas y valoradas.
